\section{Introduction}

Ultra Large-Scale Systems, Internet of Things applications, and Cyber-Physical Systems face increased resource variability compared to programs of the past, with real-world physical consequences.  Resources may be explicit: e.g., CPU, memory, or storage; or they may be implicit, e.g., libraries and protocols. The inability of software to handle resource changes effectively and automatically can produce inferior or potentially vulnerable systems~\cite{hughes2016building}. 

Self-Adaptive Software Systems (SAS) modify their own behavior in response to changes in operating conditions. Resource Adaptive Software Systems (RASS) are a subset of SAS where the trigger for adaptation is variability or unavailability of resources. For longevity and survivability of modern systems, self-adaptation for resource changes is essential hence the problem of self-adaptation is well studied~\cite{seams2018keynote}. To this end, DARPA started BRASS (Building Resource Adaptive Software Systems) program, a major initiative to bring researchers and practitioners together to solve the problem of building resource adaptive software. 

Researchers have proposed many techniques, approaches, and tools to build SAS automatically~\cite{hughes2016building,salehie2009selfadaptive,krupitzer2015a}. We previously proposed a solution called Test-Based Software Minimization to build SAS to overcome some of the limitations of previous approaches based on our work with Raytheon developers as part of DARPA BRASS program~\cite{christi2017saso}. In this work, we applied TBSM to produce adaptations of a mission-critical military system and the popular NetBeans \texttt{Java} IDE.  Based on our observations: 
\begin{enumerate}
\item {We demonstrate that unlike most other previous approaches, resource adaptation via TBSM is a conceptually simple, usable, and applicable technique. }
\item {We present challenges we faced while attempting to produce accurate resource adaptations automatically. }
\item {We present the solutions we employed to solve some of those challenges. }
\item {We explore a synergy between some of the solutions to the problems of TBSM and the problems of reliable software development.}
\end{enumerate}

In the process, we identify opportunities to further aid developers in effectively employing test-based minimization to achieve resource adaptations. 





