\section{Ongoing Work and Future Directions}
As part of DARPA BRASS project, we work with multiple software development teams and other collaborators. Our communication with these stakeholders as well as the feedback that we received from the development teams utilizing our work are driving our ongoing work and future research direction. 
\subsection{Ongoing Work}
\subsubsection{Improving Applicability}
The underlying tool hddRASS that drives TBSM is written for \texttt{Java} program only. Raytheon development team utilizes TBSM for TSAS, an application written in \texttt{Java}. We also developed a \texttt{C++} version of hddRASS that is intended to be used by ROS (Robotics Operating System) application developers to adapt against ROS version changes and other ROS package changes.  Developers can use it to build any different kind of resource adaptation for \texttt{C++} applications by providing correct test labeling. We plan to publish it soon. 

\subsubsection{Minimization vs Modification}
The primary concern that any development team might raise before evaluating TBSM to build resource adaptation is: It only offers minimization(reduction). When the software system adapts itself, the adaptation manifests itself in 3 ways (1) reduction (2) replacement (3) enhancement~\cite{hughes2016building}. The published TBSM version only supports reduction. While developing a \texttt{C++} version of hddRASS, we incorporated many other modification operators studied and used by APR to fix the fault, making few replacement capabilities available~\cite{Forrest2009genetic,Arcuri2009phdthesis,Debroy2010using}. We plan to continue to improve hddRASS to provide more modification capabilities. 

\subsection{Future Direction}
We are currently investigating multiple other ways to speed up TBSM, including using static and dynamic analysis to precompute the effects of program statements on test oracles and using test case selection and prioritization to reduce the running time of tests in TBSM’s generate-and-validate loop.

The original work in TBSM emphasized the need for a “good” test suite. Previous heuristics suggested coverage as a way to define “goodness” of a test suite; the CBLS heuristic depends on coverage information. Because of the success of AdFL heuristics in isolating and prioritizing modification targets, we plan to consider test suite diagnosability metrics as a more refined way to define the “goodness” of a test suite for TBSM, using approaches proposed by Baudry et al. and Perez et al.~\cite{Baudry2006improving,perez2017diagnosibility}

For our work, we mostly utilized existing test suites provided by the developers. For such test suites, we observe that sometimes a test that developer labeled as pertaining to a particular feature may contain code that exercises other features also. Similarly, an unlabeled test, apart from testing functionality that needs to be retained, may exercise sacrificial feature.  Availability of such tests makes it harder for TBSM to differentiate between an adaptation related modification and a pure accidental modification, resulting in underfitting or overfitting. To mitigate the situation, we plan to utilize (1) Test-case purification to segregate tests by features~\cite{xuan2014test} (2) Test-case reduction using delta debugger to remove irrelevant features from tests~\cite{stvrcausereduce,christi18reduce}.  
