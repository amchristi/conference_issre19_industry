\section{Case Study}
Building an adaptive software system is not (yet) a common practice. The example benchmarks available as part of self-adaptive exemplars are unfortunately not applicable to TBSM, due to the nature of their tests and the resources adapted~\cite{exemplars}. Hence, we rely on two real-world case studies.   

We worked with a group of developers attempting to build adaptations for a mission-critical system called the Tactical Situational Awareness System (TSAS). To focus our discussion, we present one adaptation scenario, the \textit{Elevation API} scenario. The resource under consideration is a library (an implicit resource) that has been updated to a new version. Developers labeled tests in terms of the feature tested. Certain features were marked as sacrificial (abel to be sacrificed) because the new version of the library provides functionality that was originally part of the TSAS code, the functionality these features was intended to provide. The sandboxed TSAS version that we used consists of 70 \texttt{Java} files, with 5,571 LOC, and developers labeled 5 tests as representing sacrificial functionality; the remaining tests were considered unlabeled. The developers used TBSM to build an adaptation.

For our next case study, we used the popular \textit{NetBeans IDE} for \texttt{Java}. We discussed the \textit{NetBeans IDE} case study in detail in our original work~\cite{christi2017saso}. The resource under consideration in this case was memory. To save memory, we adapted the IDE by disabling undo-redo functionality, which is potentially memory-intensive. Developers working in a memory-limited setting would prefer to lose undo-redo rather than face constant crashes due to memory exhaustion. Our target for adaptation is the \texttt{openide.awt} module, which consists of 69 \texttt{Java} files, 11,284 LOC, and 146 tests. After carefully studying the code and tests we labeled 3 tests as undo-redo related. We then applied TBSM to build a memory adaptive \textit{NetBeans IDE}. As most of the 69 \texttt{Java} files are clearly not related to the undo-redo feature, we chose the two most likely files as reduction targets to reduce processing time~\cite{christi2017saso}. 
